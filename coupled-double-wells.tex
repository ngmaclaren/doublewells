\documentclass[11pt, letterpaper]{article}

\usepackage{mathtools}
\usepackage{graphicx}

\title{Initial Experiments with the Double-Well System}
\author{Neil MacLaren}
\date{\today}

\begin{document}

\maketitle

\section{The System}

I have written two functions that implement a single double-well system, optionally with added noise and stress, and a system of coupled double-wells that interact according to an adjacency matrix.

\subsection{A Single Double-Well System}

\begin{equation}
  \Delta x = (-(x - r_1)(x - r_2)(x - r_3) + u + \varepsilon)\Delta t
\end{equation}

\begin{equation}
  x_{t+1} = x_t + \Delta x
\end{equation}

\noindent where $r_1 = 1$, $r_2 = 2$, and $r_3 = 5$ are parameters that influence the ``shape'' the double-well system, $u = 10$ is a constant representing stress being added to the system, and $\varepsilon$ is currently implemented as Gaussian noise ($\sim N(0, 10)$). 

\subsection{Coupled Double-Wells}

\begin{equation}
  \Delta x_i = (-(x_i - r_1)(x_i - r_2)(x_i - r_3) + D \sum_{j=1}^n Ax_j + u_i + \varepsilon_i)\Delta t
\end{equation}

\noindent with symbols as before and $D \in (0, 1)$  a coupling strength and $A$ an adjacency matrix.

I am focusing on a few particular triads: the complete triad, the ``feed-forward'' loop or transitive triad, the ``feed-back'' loop or intransitive triad, and a few variations on the line motif.

\section{Initial Results}

Taking first the complete triad

\includegraphics[height = 3in]{./img/complete-triad.pdf}

\noindent I vary $D$, which influences all nodes,

\includegraphics[width = \textwidth]{./img/varying-D-complete-triad.pdf}

\noindent and $u$. For $u$, I only add stress $u_1$ to node $x_1$.

\includegraphics[width = \textwidth]{./img/varying-u-completetriad.pdf}

For both, allowing a longer simulation run time allows systems with lower levels of the control parameter to break over the $r_2$ threshold. However, the systems remain qualitatively different.






\end{document}
